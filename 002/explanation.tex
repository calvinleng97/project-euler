\documentclass[12pt]{article}
 \usepackage[margin=1in]{geometry} 
\usepackage{amsmath,amsthm,amssymb,amsfonts,bbm, physics}
\usepackage{graphicx}
\usepackage{systeme}
\newcommand{\N}{\mathbb{N}}
\newcommand{\Z}{\mathbb{Z}}
\newcommand{\lm}{\lambda}
\newcommand{\I}{\mathbbm{1}}
\newtheorem{theorem}{Theorem}[section]
\newtheorem{corollary}{Corollary}[theorem]
\newtheorem{lemma}[theorem]{Lemma}
\newtheorem{proposition}[theorem]{Proposition}
\newenvironment{problem}[2][Problem]{\begin{trivlist}
\item[\hskip \labelsep {\bfseries #1}\hskip \labelsep {\bfseries #2.}]}{\end{trivlist}}

\begin{document}
 
\noindent
\textbf{Time Complexity:} $O(\log N)$

\noindent
This was solved by a simple brute force method. The only optimization was computing the Fibonnaci numbers
in the loop style, rather than the recursive style that they warn of in computer science classes. The time
complexity is $O(\log N)$ because the Fibonnaci numbers grow exponentially. This can be seen from the formula
that gives the $n$-th Fibonnaci number, which is
\begin{align*}
  F(n) = \frac{\phi^n - (-\phi)^{-n}}{\sqrt{5}}
\end{align*}

\noindent
where $\phi = \frac{1 + \sqrt{5}}{2}$.

\end{document}
