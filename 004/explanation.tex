\documentclass[12pt]{article}
 \usepackage[margin=1in]{geometry} 
\usepackage{amsmath,amsthm,amssymb,amsfonts,bbm, physics}
\usepackage{graphicx}
\usepackage{systeme}
\newcommand{\N}{\mathbb{N}}
\newcommand{\Z}{\mathbb{Z}}
\newcommand{\lm}{\lambda}
\newcommand{\I}{\mathbbm{1}}
\newtheorem{theorem}{Theorem}[section]
\newtheorem{corollary}{Corollary}[theorem]
\newtheorem{lemma}[theorem]{Lemma}
\newtheorem{proposition}[theorem]{Proposition}
\newenvironment{problem}[2][Problem]{\begin{trivlist}
\item[\hskip \labelsep {\bfseries #1}\hskip \labelsep {\bfseries #2.}]}{\end{trivlist}}

\begin{document}
 
\noindent
\textbf{Time Complexity:} $O(de^d )$ where $d$ is the number of digits

\noindent
This was solved by a simple brute force method. The time complexity is $O(de^d)$ because the number of products we have to test is exponential with respect to the number of digits, and for each product we have to perform $O(d)$ operations, as the length of the string to test for palindrome-ness is of length $d$.

\end{document}
