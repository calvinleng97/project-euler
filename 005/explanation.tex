\documentclass[12pt]{article}
 \usepackage[margin=1in]{geometry} 
\usepackage{amsmath,amsthm,amssymb,amsfonts,bbm, physics}
\usepackage{graphicx}
\usepackage{systeme}
\newcommand{\N}{\mathbb{N}}
\newcommand{\Z}{\mathbb{Z}}
\newcommand{\lm}{\lambda}
\newcommand{\I}{\mathbbm{1}}
\newtheorem{theorem}{Theorem}[section]
\newtheorem{corollary}{Corollary}[theorem]
\newtheorem{lemma}[theorem]{Lemma}
\newtheorem{proposition}[theorem]{Proposition}
\newenvironment{problem}[2][Problem]{\begin{trivlist}
\item[\hskip \labelsep {\bfseries #1}\hskip \labelsep {\bfseries #2.}]}{\end{trivlist}}

\begin{document}

\noindent
This problem was done by hand. One can see that the desired answer is simply $lcm(1, 2, \dots, 20)$. 
To compute this, I simply found the factorizations of each number combined them all, removing repeats.
So, I ended up with a product that was 1 * 2 * 3 * 2 * 5 * 7 * 2 * 3 * 11 * 13 * 2 * 17 * 19, which 
gives us the desired answer.

\end{document}
