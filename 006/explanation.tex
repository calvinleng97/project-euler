\documentclass[12pt]{article}
 \usepackage[margin=1in]{geometry} 
\usepackage{amsmath,amsthm,amssymb,amsfonts,bbm, physics}
\usepackage{graphicx}
\usepackage{systeme}
\newcommand{\N}{\mathbb{N}}
\newcommand{\Z}{\mathbb{Z}}
\newcommand{\lm}{\lambda}
\newcommand{\I}{\mathbbm{1}}
\newtheorem{theorem}{Theorem}[section]
\newtheorem{corollary}{Corollary}[theorem]
\newtheorem{lemma}[theorem]{Lemma}
\newtheorem{proposition}[theorem]{Proposition}
\newenvironment{problem}[2][Problem]{\begin{trivlist}
\item[\hskip \labelsep {\bfseries #1}\hskip \labelsep {\bfseries #2.}]}{\end{trivlist}}

\begin{document}

\noindent
This problem was done by hand. Simply use the sum of squares formula and the sum of consecutive integers formula. Namely,

\begin{align*}
    &\sum_{i = 1}^n i = \frac{n(n+1)}{2}\\
    &\sum_{i = 1}^n i^2 = \frac{n(n+1)(2n+1)}{6}.
\end{align*}
\end{document}
