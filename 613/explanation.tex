\documentclass[12pt]{article}
 \usepackage[margin=1in]{geometry} 
\usepackage{amsmath,amsthm,amssymb,amsfonts,bbm, physics}
\usepackage{graphicx}
\usepackage{systeme}
\newcommand{\N}{\mathbb{N}}
\newcommand{\Z}{\mathbb{Z}}
\newcommand{\lm}{\lambda}
\newcommand{\I}{\mathbbm{1}}
\newtheorem{theorem}{Theorem}[section]
\newtheorem{corollary}{Corollary}[theorem]
\newtheorem{lemma}[theorem]{Lemma}
\newtheorem{proposition}[theorem]{Proposition}
\newenvironment{problem}[2][Problem]{\begin{trivlist}
\item[\hskip \labelsep {\bfseries #1}\hskip \labelsep {\bfseries #2.}]}{\end{trivlist}}

\begin{document}
 
\noindent
This problem was done by hand. We represent the triangle on the plane where the origin is at the right angle of the triangle and extends 40 in the $X$ direction and 30 in the $Y$ direction. We attack the problem by first uniformly choosing a random point $P$ on the plane. By symmetry that will be seen later in the problem, we can instead uniformly choose a point on the 30 by 40 rectangle. Therefore, if $X$ is the $X$-component of $P$ and $Y$ is the $Y$-component, we have
\begin{align*}
    &X \sim \text{Uniform}(0, 40),\\
    &Y \sim \text{Uniform}(0, 30).
\end{align*}

\noindent
Denote the vectors extending from the random point to the non-origin corners of the triangle as $\textbf{u}$ and $\textbf{v}$. We have
\begin{align*}
    \textbf{u} = (-X, 30 - Y),\\
    \textbf{v} = (40 - X, -Y).
\end{align*}

\noindent
We can compute the angle $\alpha$ that the three points form at point $P$ from the dot product, arriving at
\begin{align*}
    \alpha &= \arccos\left(\frac{\textbf{u} \cdot \textbf{v}}{|\textbf{u}| |\textbf{v}|}\right)\\
    &= \arccos\left(\frac{X^2 - 40X + Y^2 - 30Y}{\sqrt{(X^2 + Y^2 - 60Y + 900)(X^2 - 800X + 1600 + Y^2)}}\right)\\
    &:= f(X, Y).
\end{align*}

\noindent
Now that we have $\alpha = f(X, Y)$, we see that 
\begin{align*}
    \mathbb{P}(\text{ant leaves longest side}) &= \frac{\mathbb{E}[\alpha]}{2\pi}\\
    &= \int_0^{30}\int_0^{40} \frac{f(X, Y)}{2400\pi} \text{dX}\text{dY},
\end{align*}

\noindent
which we compute through online definite integral calculators to arrive at

\begin{align*}
    \mathbb{P}(\text{ant leaves longest side}) = 0.3916721504...
\end{align*}

\noindent
Computation of the integral can be sped up by reducing the triangle to a 3-4-5 triangle and to compute complementary angles formed using $\arctan$ which will let us arrive at a closed form solution. However, the solution given is a working brute force computation.
\end{document}
